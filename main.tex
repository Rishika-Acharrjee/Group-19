\documentclass{article}
\pagestyle{empty}
\usepackage{calc}
\usepackage{graphicx}
\usepackage{float}
\usepackage{eso-pic}

\newlength{\PageFrameTopMargin}
\newlength{\PageFrameBottomMargin}
\newlength{\PageFrameLeftMargin}
\newlength{\PageFrameRightMargin}

\setlength{\PageFrameTopMargin}{2cm}
\setlength{\PageFrameBottomMargin}{2cm}
\setlength{\PageFrameLeftMargin}{2cm}
\setlength{\PageFrameRightMargin}{2cm}

\makeatletter

\newlength{\Page@FrameHeight}
\newlength{\Page@FrameWidth}

\AddToShipoutPicture{
  \thinlines
  \setlength{\Page@FrameHeight}{\paperheight-\PageFrameTopMargin-\PageFrameBottomMargin}
  \setlength{\Page@FrameWidth}{\paperwidth-\PageFrameLeftMargin-\PageFrameRightMargin}
  \put(\strip@pt\PageFrameLeftMargin,\strip@pt\PageFrameTopMargin){
    \framebox(\strip@pt\Page@FrameWidth, \strip@pt\Page@FrameHeight){}}}

\makeatother
\begin{document}

\begin{figure}[h]
    \includegraphics[width=1\textwidth]{Line.png}
\end{figure}

\begin{figure}[h]
\begin{center}
    \includegraphics[width=0.5\textwidth]{MAKAUT_LOGO}
\end{center}
\end{figure}
\vspace{1cm}
\begin{center}
     {\Huge Lab Notebook}
\end{center}
\begin{center}
    {\Large
    Software Tools and Technology Lab}
\end{center}
\begin{center}
        {\Large SEBCA1191}
\end{center}
\vspace{1cm}
\renewcommand{\arraystretch}{2}
\hspace*{0.08in}
\begin{tabular}{ |c|c|c| }
\hline
\multicolumn{3}{|c|}{\Large \textbf{\textit{Group 19}}} \\
\hline
NAME & ROLL NO.& DEPARTMENT \\
\hline
Rishika Acharjee [Group-Leader]  & 30001223016 & BCA \\
\hline
Ankita Ghoshal & 330085323012 & Bsc IT in CS \\
\hline
Moumita Modak  & 30054623014 & Bsc IT in AI \\
\hline
Pritam Sarkar & 30001223042 & BCA \\
\hline
Titli Biswas & 30059223018 & BSc in Data Science \\
\hline
\end{tabular}
\vspace{1cm}
\begin{center}
    \includegraphics[width=1\textwidth]{Line.png}
\end{center}
\newpage
\newpage
\begin{center}
    \section*{\textcolor{blue}{\textbf{Index}}}
\end{center}

\begin{table}[h!]
\centering
\begin{tabular}{|c|p{10cm}|}
    \hline
    \textbf{Serial No.} & \textbf{Questions} \\ \hline
    1 & Calculator Program using C \\ \hline
    2 & Question 2 description here \\ \hline
    3 & Question 3 description here \\ \hline
    4 & Question 4 description here \\ \hline
    5 & Question 5 description here \\ \hline
\end{tabular}
\end{table}


\newpage
\begin{center}
    \textbf{\LARGE{{Member 2: [Ankita Ghoshal]}}}
\end{center}
\section{Lab 1: Calculator Program using C}
\subsection{Objective}
The objective of this lab is to develop a basic calculator program using the C programming language. The calculator will perform simple arithmetic operations like addition, subtraction, multiplication, and division based on user input.

\subsection{Program Overview}
The calculator program is designed to:
\begin{itemize}
    \item Accept two numbers from the user.
    \item Prompt the user to select an arithmetic operation (Addition, Subtraction, Multiplication, Division).
    \item Perform the selected operation.
    \item Display the result of the operation to the user.
\end{itemize}

The program includes error handling to manage division by zero and other invalid inputs.

\subsection{Code Implementation}
The following is the C code for the calculator program:

\begin{verbatim}
#include <stdio.h>

int main() {
    char operator;
    double num1, num2, result;

    printf("Enter an operator (+, -, *, /): ");
    scanf("%c", &operator);

    printf("Enter two operands: ");
    scanf("%lf %lf", &num1, &num2);

    switch(operator) {
        case '+':
            result = num1 + num2;
            break;
        case '-':
            result = num1 - num2;
            break;
        case '*':
            result = num1 * num2;
            break;
        case '/':
            if (num2 != 0)
                result = num1 / num2;
            else {
                printf("Error! Division by zero.\n");
                return -1;
            }
            break;
        default:
            printf("Error! Operator is not correct\n");
            return -1;
    }

    printf("Result: %.2lf\n", result);
    return 0;
}
\end{verbatim}

\subsection{Compiling and Running the Program}
To compile and run the calculator program:
\begin{enumerate}
    \item Open a terminal or command prompt.
    \item Navigate to the directory where the C file is located.
    \item Compile the program using a C compiler (e.g., GCC):
    \begin{verbatim}
    gcc calculator.c -o calculator
    \end{verbatim}
    \item Run the compiled program:
    \begin{verbatim}
    ./calculator
    \end{verbatim}
\end{enumerate}

\subsection{Adding the Calculator Program to GitHub Repository}
To add this calculator program to a GitHub repository, follow these steps:

\subsubsection{Step 1: Initialize a Local Git Repository}
\begin{enumerate}
    \item Open the terminal and navigate to the directory where your \texttt{calculator.c} file is located.
    \item If you haven't already, initialize a Git repository in that directory:
    \begin{verbatim}
    git init
    \end{verbatim}
    This command creates a new Git repository in the current directory.
\end{enumerate}

\subsubsection{Step 2: Add the File to the Repository}
\begin{enumerate}
    \item Add the \texttt{calculator.c} file to the staging area:
    \begin{verbatim}
    git add calculator.c
    \end{verbatim}
    This command stages the file, indicating that you want to include it in the next commit.
\end{enumerate}

\subsubsection{Step 3: Commit the Changes}
\begin{enumerate}
    \item Commit the file to the repository with a meaningful message:
    \begin{verbatim}
    git commit -m "Add calculator program in C"
    \end{verbatim}
\end{enumerate}

\subsubsection{Step 4: Push the Changes to GitHub}
\begin{enumerate}
    \item Link your local repository to a remote GitHub repository:
    \begin{verbatim}
    git remote add origin https://github.com/yourusername/your-repo-name.git
    \end{verbatim}
    \item Push the changes to the GitHub repository:
    \begin{verbatim}
    git push -u origin master
    \end{verbatim}
\end{enumerate}

\subsubsection{Step 5: Verify the Upload}
\begin{enumerate}
    \item Go to your GitHub repository URL in a web browser.
    \item Verify that the \texttt{calculator.c} file is listed and accessible in the repository.
\end{enumerate}
\end{document}

\section*{Latex Assignment by Pabitra Sir}

\subsection*{Entry by Titli Biswas}
\subsection{Mathematical Notation}


\begin{enumerate}
    \item Superscripts, subscripts, and Greek letters
    \begin{itemize}
        \item $2224$
        \item $22_{24}$
        \item $2224^{\pi}$
        \item $\cos \theta$
        \item $\tan^{-1}(2.224)$
        \item $\log_{22}24$
        \item $\ln 2224$
        \item $e^{2.224}$
        \item $0 < x \leq 2224$
        \item $y \geq 2224$
    \end{itemize}

    \item Roots, fractions, and displaystyle
    \begin{itemize}
        \item $\sqrt{2224}$
        \item $\sqrt[22]{24}$
        \item Normal: $\frac{22}{24}$ Displaystyle: $\displaystyle\frac{22}{24}$
        \item Normal: $\frac{2}{2+ \frac{2}{4}}$ Displaystyle: $\displaystyle\frac{2}{2+ \frac{2}{4}}$
        \item Normal: $\sqrt{\frac{22}{24}}$ Displaystyle: $\displaystyle\sqrt{\frac{22}{24}}$
    \end{itemize}

    \item Delimiters
    \begin{itemize}
        \item Display math mode: $\displaystyle\left(2 + \frac{2}{24}\right)$
        \item Display math mode: $\displaystyle\left|\frac{22 - 2}{4}\right|$
    \end{itemize}

    \item Tables and equation arrays
    \begin{itemize}
        \item
        \[
        \begin{array}{c|c|c|c|c}
        x & 1 & 2 & 3 & 4 \\
        f(x) & 2 & 2 & 2 & 2 \\
        \end{array}
        \]
        \item
        \begin{align}
        2 + 2 - 2 \times 4 &= x \\
        2 + 2 - 8 &= x \\
        4 - 6 &= x \\
        x &= -2
        \end{align}
    \end{itemize}

    \item Functions \& Formulas
    \begin{itemize}
        \item The quadratic formula:
        \[
        x = \frac{-b \pm \sqrt{b^2 - 4ac}}{2a}
        \]
        \item The function $f(x) = \left(\frac{x+2}{2}\right)^2 - \frac{2}{4}$ has domain $D_f : (-\infty,\infty)$ and range $R_f : \left[\frac{-7}{2},\infty\right)$.
        \item Definition of a Derivative: $\lim\limits_{h \to 0} \frac{f(x+h)-f(x)}{h} = f'(x)$
        \item Chain Rule: $\left[f(g(x))\right]' = f'(g(x)) \cdot g'(x)$
        \item $\frac{d^2y}{dx^2} = f''(x)$
        \item $\int \sec^2 x \,dx = \tan x + C$
        \item $\int e^{2x} \,dx = \frac{1}{2}e^{2x} + C$
        \item Fundamental Theorem of Calculus, Part 1: $\int_a^b f'(x) \,dx = f(b) - f(a)$
        \item Fundamental Theorem of Calculus, Part 2: $\frac{d}{dx} \left( \int_a^{g(x)} f(t) \,dt \right) = f(g(x)) \cdot g'(x)$
        \item Euler's Method: $y_1 = y_0 + hF(x_0, y_0)$ where $h$ is the step size, and $F(x, y) = \frac{dy}{dx}$
        \item $a_n = \{2224, \frac{2224}{2}, \frac{2224}{4}, \frac{2224}{8}, \cdots, \frac{2224}{2^n} \}$ represents a geometric sequence.
        \item $S_n = \sum_{n=1}^{\infty} \frac{2224}{2^n}$ is a convergent geometric series since $\left|r\right| = \left|\frac{1}{2}\right| < 1$.
        \item Taylor Series: $\sum_{n=0}^{\infty} \frac{f^{(n)}(c)}{n!} (x - c)^n$
        \item Velocity Vector: $\vec{v}(t) = x'(t)\vec{i} + y'(t)\vec{j} = \left\langle \frac{dx}{dt}, \frac{dy}{dt} \right\rangle$
        \item Area of Polar Curve: $A = \frac{1}{2} \int_{\alpha}^{\beta} r^2 \,d\theta$
    \end{itemize}
\end{enumerate}

\end{document}
