\documentclass{article}
\pagestyle{empty}
\usepackage{calc}
\usepackage{graphicx}
\usepackage{float}
\usepackage{eso-pic}
\usepackage{url}
\usepackage{hyperref}
\usepackage{listings}
\usepackage{xcolor}
\usepackage[utf8]{inputenc}
\usepackage{titlesec}
\usepackage{fancyhdr}
\usepackage{hyperref}
\usepackage{fontawesome}
\usepackage{enumitem}
\usepackage{geometry}
\usepackage{amsmath}
\newlength{\PageFrameTopMargin}
\newlength{\PageFrameBottomMargin}
\newlength{\PageFrameLeftMargin}
\newlength{\PageFrameRightMargin}
\setlength{\PageFrameTopMargin}{2cm}
\setlength{\PageFrameBottomMargin}{2cm}
\setlength{\PageFrameLeftMargin}{2cm}
\setlength{\PageFrameRightMargin}{2cm}
\makeatletter
\newlength{\Page@FrameHeight}
\newlength{\Page@FrameWidth}
\AddToShipoutPicture{
  \thinlines
  \setlength{\Page@FrameHeight}{\paperheight-\PageFrameTopMargin-\PageFrameBottomMargin}
  \setlength{\Page@FrameWidth}{\paperwidth-\PageFrameLeftMargin-\PageFrameRightMargin}
  \put(\strip@pt\PageFrameLeftMargin,\strip@pt\PageFrameTopMargin){
    \framebox(\strip@pt\Page@FrameWidth, \strip@pt\Page@FrameHeight){}}}

\makeatother
\begin{document}
\begin{figure}[h]
    \includegraphics[width=1\textwidth]{Line.png}
\end{figure}

\begin{figure}[h]
\begin{center}
    \includegraphics[width=0.5\textwidth]{MAKAUT_LOGO}
\end{center}
\end{figure}
\vspace{1cm}
\begin{center}
     {\Huge Lab Notebook}
\end{center}
\begin{center}
    {\Large
    Software Tools and Technology Lab}
\end{center}
\begin{center}
        {\Large SEBCA1191}
\end{center}
\vspace{1cm}
\renewcommand{\arraystretch}{2}
\begin{center}
\begin{tabular}{ |c|c|c| }
\hline
\multicolumn{3}{|c|}{\Large \textbf{\textit{Group 19}}} \\
\hline
NAME & ROLL NO.& DEPARTMENT \\
\hline
Rishika Acharjee [Group-Leader]  & 30001223016 & BCA \\
\hline
Ankita Ghoshal & 330085323012 & Bsc IT in CS \\
\hline
Pritam Sarkar  & 30001223042 & BCA \\
\hline
Titli Biswas & 30059223018 & BSc in Forensic Science \\
\hline
Moumita Modak & 30054623014 & Bsc IT in AI \\
\hline
\end{tabular}
\end{center}
\lstset{
    basicstyle=\ttfamily,
    keywordstyle=\color{blue},
    commentstyle=\color{gray},
    stringstyle=\color{purple},
    breaklines=true,
}
\newpage
\begin{center}
    \Huge\textbf{Index}
\end{center}

\vspace{1cm}

\begin{tabular}{|c|p{10cm}|c|}
    \hline
    \textbf{Sl. No.} & \textbf{Experiment} & \textbf{Page No.} \\
    \hline
    1 & Introduction to Latex & 1\\
    \hline
    2 & Calculator Program using C & 3\\
    \hline
    3 & Converting a submit button to Chin Tapak Dum Dum & 7 \\
    \hline
    4 & Mathematical Notation & 10\\
    \hline
    5 & CV & 12 \\
    \hline

\end{tabular}
\newpage

\begin{titlepage}
    \centering
    \vspace*{\fill}
    
    \Huge \textbf{Acknowledgment}
    
    \vspace{1cm}
    
    \large
    We gratefully acknowledge the efforts of all group members in contributing their respective lab notebook entries and committing them to the GitHub repository. We extend our sincere gratitude to the subject-faculties for their valuable guidance and support throughout this assignment. This experience has significantly enhanced our skills in LaTeX and collaborative work.
    
    \vspace*{\fill}
    
    \begin{flushright}
    
        \vspace{2cm}
        \underline{\hspace{3.7cm}}
        \\
        Student's Signature
    \end{flushright}
\end{titlepage}

\newpage
\pagestyle{plain}
\fancyhf{}
\begin{center}
    \textbf{\LARGE{{Group-Leader: [Rishika Acharjee]}}}
\end{center}
\section{Introduction to \LaTeX} is a typesetting system that is widely used for producing scientific and technical documents. It is especially popular for creating documents with complex mathematical formulas, tables, and figures. Unlike word processors, \LaTeX{} provides greater control over document structure and presentation, making it a preferred tool in academia and industry.

\subsection {Key Features of \LaTeX}
Some of the key features of \LaTeX{} include:
\begin{itemize}
    \item High-quality typesetting, especially for mathematical and technical content.
    \item Separation of content and style, allowing users to focus on writing.
    \item Automated table of contents, bibliography, and cross-references.
    \item Supports a wide range of document types, including articles, books, reports, and presentations.
    \item Extensible through packages for additional functionality, such as graphics, tables, and advanced formatting.
\end{itemize}
\subsection{Why Use \LaTeX?}
There are several reasons to use \LaTeX{}:
\begin{itemize}
    \item \textbf{Precision}: \LaTeX{} provides exceptional control over document structure and formatting, ensuring that everything looks just right.
    \item \textbf{Mathematical Formulas}: It is the standard for creating documents that contain mathematical symbols and equations.
    \item \textbf{Professional Quality}: Documents created with \LaTeX{} have a professional appearance, suitable for academic papers, theses, and books.
    \item \textbf{Consistency}: Once a style is defined, it is applied consistently throughout the document.
    \item \textbf{Collaboration}: Multiple users can work on the same document without formatting issues, as \LaTeX{} is plain text-based.
\end{itemize}
\subsection{Creating a \LaTeX{} Repository on GitHub using GitHub Desktop}
Version control is crucial for managing changes in documents, especially when working on complex projects. GitHub provides a platform for managing repositories, and GitHub Desktop simplifies working with GitHub.

\begin{center}
To create a \LaTeX{} repository on GitHub using GitHub Desktop:
\end{center}
\begin{enumerate}
    \item \textbf{Install GitHub Desktop}: Download and install GitHub Desktop from \url{https://desktop.github.com/}.
    \item \textbf{Create a GitHub Account}: If you don’t have one, sign up for GitHub at \url{https://github.com/}.
    \item \textbf{Create a New Repository}: Open GitHub Desktop and click on \textit{File} $\rightarrow$ \textit{New Repository}. Name your repository, select the local path, and ensure the repository is initialized with a README.
    \item \textbf{Clone the Repository}: After creating the repository on GitHub, use GitHub Desktop to clone the repository to your local machine by clicking on \textit{File} $\rightarrow$ \textit{Clone Repository}.
    \item \textbf{Add Your \LaTeX{} Files}: Open the cloned folder on your computer and add your \LaTeX{} source files (.tex, .bib, etc.). 
    \item \textbf{Commit and Push}: After adding your files, return to GitHub Desktop, write a commit message, and click on \textit{Commit to main}. Then click on \textit{Push origin} to upload your changes to GitHub.
\end{enumerate}
\begin{figure}[h!]
    \centering
    \includegraphics[width=0.5\linewidth]{LaTeX_logo.png}
    \caption{\LaTeX}
    
\end{figure}
\newpage
\begin{center}
    \textbf{\LARGE{{Member 2: [Ankita Ghoshal]}}}
\end{center}
\section{Calculator Program using C}
\subsection{Objective}
The objective of this lab is to develop a basic calculator program using the C programming language. The calculator will perform simple arithmetic operations like addition, subtraction, multiplication, and division based on user input.

\subsection{Program Overview}
The calculator program is designed to:
\begin{itemize}
    \item Accept two numbers from the user.
    \item Prompt the user to select an arithmetic operation (Addition, Subtraction, Multiplication, Division).
    \item Perform the selected operation.
    \item Display the result of the operation to the user.
\end{itemize}

The program includes error handling to manage division by zero and other invalid inputs.

\subsection{Code Implementation}
The following is the C code for the calculator program:

\begin{verbatim}
#include <stdio.h>

int main() {
    char operator;
    double num1, num2, result;

    printf("Enter an operator (+, -, *, /): ");
    scanf("%c", &operator);

    printf("Enter two operands: ");
    scanf("%lf %lf", &num1, &num2);

    switch(operator) {
        case '+':
            result = num1 + num2;
            break;
        case '-':
            result = num1 - num2;
            break;
        case '*':
            result = num1 * num2;
            break;
        case '/':
            if (num2 != 0)
                result = num1 / num2;
            else {
                printf("Error! Division by zero.\n");
                return -1;
            }
            break;
        default:
            printf("Error! Operator is not correct\n");
            return -1;
    }

    printf("Result: %.2lf\n", result);
    return 0;
}
\end{verbatim}


\subsection{Compiling and Running the Program}
To compile and run the calculator program:
\begin{enumerate}
    \item Open a terminal or command prompt.
    \item Navigate to the directory where the C file is located.
    \item Compile the program using a C compiler (e.g., GCC):
    \begin{verbatim}
    gcc calculator.c -o calculator
    \end{verbatim}
    \item Run the compiled program:
    \begin{verbatim}
    ./calculator
    \end{verbatim}
\end{enumerate}

\subsection{Adding the Calculator Program to GitHub Repository}
To add this calculator program to a GitHub repository, follow these steps:

\subsubsection{Step 1: Initialize a Local Git Repository}
\begin{enumerate}
    \item Open the terminal and navigate to the directory where your \texttt{calculator.c} file is located.
    \item If you haven't already, initialize a Git repository in that directory:
    \begin{verbatim}
    git init
    \end{verbatim}
    This command creates a new Git repository in the current directory.
\end{enumerate}

\subsubsection{Step 2: Add the File to the Repository}
\begin{enumerate}
    \item Add the \texttt{calculator.c} file to the staging area:
    \begin{verbatim}
    git add calculator.c
    \end{verbatim}
    This command stages the file, indicating that you want to include it in the next commit.
\end{enumerate}

\subsubsection{Step 3: Commit the Changes}
\begin{enumerate}
    \item Commit the file to the repository with a meaningful message:
    \begin{verbatim}
    git commit -m "Add calculator program in C"
    \end{verbatim}
\end{enumerate}

\subsubsection{Step 4: Push the Changes to GitHub}
\begin{enumerate}
    \item Link your local repository to a remote GitHub repository:
    \begin{verbatim}
    git remote add origin https://github.com/yourusername/your-repo-name.git
    \end{verbatim}
    \item Push the changes to the GitHub repository:
    \begin{verbatim}
    git push -u origin master
    \end{verbatim}
\end{enumerate}

\subsubsection{Step 5: Verify the Upload}
\begin{enumerate}
    \item Go to your GitHub repository URL in a web browser.
    \item Verify that the \texttt{calculator.c} file is listed and accessible in the repository.
\end{enumerate}
The given code takes two double inputs (\texttt{num1} and \texttt{num2}), performs a mathematical operation based on the operator provided, and prints the result. Below are the possible outputs depending on the operator and inputs.

\subsection{1. Addition Case (\texttt{'+'} Operator)}
If the operator is \texttt{'+'}, the code will add \texttt{num1} and \texttt{num2} and output the result.

\[
\text{Input:} \ 5.2, \ 3.8, \ +
\]
\[
\text{Output:} \ \text{Result:} \ 9.00
\]

\subsection{2. Subtraction Case (\texttt{'-'} Operator)}
If the operator is \texttt{'-'}, the code will subtract \texttt{num2} from \texttt{num1} and output the result.

\[
\text{Input:} \ 10.5, \ 4.2, \ -
\]
\[
\text{Output:} \ \text{Result:} \ 6.30
\]

\subsection{3. Multiplication Case (\texttt{'*'} Operator)}
If the operator is \texttt{'*'}, the code will multiply \texttt{num1} and \texttt{num2} and output the result.

\[
\text{Input:} \ 7.0, \ 3.0, \ *
\]
\[
\text{Output:} \ \text{Result:} \ 21.00
\]

\subsection{4. Division Case (\texttt{'/'} Operator)}
If the operator is \texttt{'/'} and \texttt{num2} is not zero, the code will divide \texttt{num1} by \texttt{num2} and output the result.

\[
\text{Input:} \ 20.0, \ 4.0, \ /
\]
\[
\text{Output:} \ \text{Result:} \ 5.00
\]

\subsection{5. Division by Zero Case}
If the operator is \texttt{'/'} and \texttt{num2} is zero, the code will print an error message and return \texttt{-1}.

\[
\text{Input:} \ 10.0, \ 0, \ /
\]
\[
\text{Output:} \ \text{Error! Division by zero.}
\]

\subsection{6. Invalid Operator Case}
If an invalid operator is provided, the code will print an error message and return \texttt{-1}.

\[
\text{Input:} \ 5.0, \ 3.0, \ \hat{ } 
\]
\[
\text{Output:} \ \text{Error! Operator is not correct.}
\]
\vspace{10in}
\begin{center}
    \textbf{\LARGE{{Member 3: [Pritam Sarkar]}}}
\end{center}
\section{Introduction}
This document outlines the process of modifying a "Submit" button in a mind reader application and submitting a pull request to the original GitHub repository. The repository in question is available at \url{https://github.com/GeekAyan/STT}. The modification includes renaming the button and fixing proportion issues.

\subsection{Cloning the GitHub Repository}
\textbf{Step}: Clone the GitHub repository using GitHub Desktop.

\textbf{Action}:
\begin{itemize}
    \item Open GitHub Desktop and select \texttt{File > Clone Repository}.
    \item Enter the repository URL: \url{https://github.com/GeekAyan/STT} and select a directory to clone it.
\end{itemize}

\subsection{Opening the Project in an IDE}
\textbf{Step}: Open the cloned project using your preferred IDE (e.g., VS Code, PyCharm).

\textbf{Action}:
\begin{itemize}
    \item Open the folder containing the cloned project.
    \item Review the \texttt{README.md} for instructions on how to run the project.
\end{itemize}

\subsection{Install Dependencies}
\textbf{Step}: Install any dependencies required by the project as per the \texttt{README.md} file.

\textbf{Action}:
\begin{itemize}
    \item Set up the environment. If the project uses Python, create a virtual environment and install dependencies using:
    \begin{lstlisting}[language=bash]
    pip install -r requirements.txt
    \end{lstlisting}
    \item Follow other system requirements mentioned in the \texttt{README.md}.
\end{itemize}

\subsection{Running the Application}
\textbf{Step}: Run the application as per the instructions in \texttt{README.md}.

\textbf{Action}:
\begin{itemize}
    \item Use your IDE's terminal to run the project.
    \item Ensure the application works as expected.
\end{itemize}

\subsection{Renaming the Submit Button}
\textbf{Step}: Rename the button from "Submit" to "Chin Tapak Dum Dum."

\textbf{Action}:
\begin{itemize}
    \item Find the code section responsible for the submit button's label.
    \item Modify the label. For example:
    \begin{lstlisting}[language=HTML]
    <button id="submit" name="submit">Chin Tapak Dum Dum</button>
    \end{lstlisting}
\end{itemize}

\subsection{Fixing the Button Proportion}
\textbf{Step}: After renaming the button, analyze and adjust its proportions.

\textbf{Action}:
\begin{itemize}
    \item Check the CSS properties related to the button's size, padding, and font.
    \item Modify the button’s CSS if needed, for example:
    \item Save changes and re-run the application to check the button's appearance.
\end{itemize}

\subsection{Testing the Application}
\textbf{Step}: Test the application after modifying the button.

\textbf{Action}: 
\begin{itemize}
    \item Run the application again to verify that the button looks correct and functions properly.
\end{itemize}

\subsection{Committing the Changes}
\textbf{Step}: Commit your changes locally.

\textbf{Action}:
\begin{itemize}
    \item Stage the files and commit with a descriptive message, for example:
    \begin{lstlisting}[language=bash]
    git commit -m "Renamed submit button and fixed proportion issue"
    \end{lstlisting}
\end{itemize}

\subsection{Pushing Changes to Your Fork}
\textbf{Step}: Push your changes to your GitHub fork.

\textbf{Action}:
\begin{itemize}
    \item If you haven’t forked the repository, go to the GitHub page and fork it.
    \item Add the forked repository as a remote and push your changes:
    \begin{lstlisting}[language=bash]
    git remote add origin https://github.com/<YourGitHubUsername>/STT.git
    git push origin main
    \end{lstlisting}
\end{itemize}

\subsection{Creating a Pull Request}
\textbf{Step}: Create a pull request to the original repository.

\textbf{Action}:
\begin{itemize}
    \item Go to your fork on GitHub and create a new pull request.
    \item Provide a descriptive title and details of your changes.
    \item Submit the pull request.
\end{itemize}

\subsection{Review and Approval}
\textbf{Step}: Wait for feedback or approval.
\textbf{Action}:
\begin{itemize}
    \item Respond to feedback or requested changes, if any.
    \item Once approved, your changes will be merged into the original project.
\end{itemize}
\vspace{20in}

\begin{center}
\textbf{\LARGE{{Member 4: TITLI BISWAS}}} 
\end{center}
\section{Mathematical Notation}

\begin{enumerate}
    \item Superscripts, subscripts, and Greek letters
    \begin{itemize}
        \item $2224$
        \item $22_{24}$
        \item $2224^{\pi}$
        \item $\cos \theta$
        \item $\tan^{-1}(2.224)$
        \item $\log_{22}24$
        \item $\ln 2224$
        \item $e^{2.224}$
        \item $0 < x \leq 2224$
        \item $y \geq 2224$
    \end{itemize}

    \item Roots, fractions, and displaystyle
    \begin{itemize}
        \item $\sqrt{2224}$
        \item $\sqrt[22]{24}$
        \item Normal: $\frac{22}{24}$ Displaystyle: $\displaystyle\frac{22}{24}$
        \item Normal: $\frac{2}{2+ \frac{2}{4}}$ Displaystyle: $\displaystyle\frac{2}{2+ \frac{2}{4}}$
        \item Normal: $\sqrt{\frac{22}{24}}$ Displaystyle: $\displaystyle\sqrt{\frac{22}{24}}$
    \end{itemize}

    \item Delimiters
    \begin{itemize}
        \item Display math mode: $\displaystyle\left(2 + \frac{2}{24}\right)$
        \item Display math mode: $\displaystyle\left|\frac{22 - 2}{4}\right|$
    \end{itemize}

    index\item Tables and Equation Arrays
        \begin{enumerate}
        \item
        \begin{tabular}{|c||c|c|c|c|}
        \hline
        $x$ & 1 & 2 & 3 & 4 \\
        \hline
        $f(x)$ & 1 & 1 & 2 & 3 \\
        \hline
        \end{tabular}
        \item
        \begin{align}
        1 + 1 - 2 \times 3 &= x \tag{1} \\
        1 + 1 - 6 &= x \tag{2} \\
        2 - 6 &= x \tag{3} \\
        x &= -4 \tag{4}
        \end{align}
        \end{enumerate} 

    \item Functions \& Formulas
    \begin{itemize}
        \item The quadratic formula:
        \[
        x = \frac{-b \pm \sqrt{b^2 - 4ac}}{2a}
        \]
        \item The function $f(x) = \left(\frac{x+2}{2}\right)^2 - \frac{2}{4}$ has domain $D_f : (-\infty,\infty)$ and range $R_f : \left[\frac{-7}{2},\infty\right)$.
        \item Definition of a Derivative: $\lim\limits_{h \to 0} \frac{f(x+h)-f(x)}{h} = f'(x)$
        \item Chain Rule: $\left[f(g(x))\right]' = f'(g(x)) \cdot g'(x)$
        \item $\frac{d^2y}{dx^2} = f''(x)$
        \item $\int \sec^2 x \,dx = \tan x + C$
        \item $\int e^{2x} \,dx = \frac{1}{2}e^{2x} + C$
        \item Fundamental Theorem of Calculus, Part 1: $\int_a^b f'(x) \,dx = f(b) - f(a)$
        \item Fundamental Theorem of Calculus, Part 2: $\frac{d}{dx} \left( \int_a^{g(x)} f(t) \,dt \right) = f(g(x)) \cdot g'(x)$
        \item Euler's Method: $y_1 = y_0 + hF(x_0, y_0)$ where $h$ is the step size, and $F(x, y) = \frac{dy}{dx}$
        \item $a_n = \{2224, \frac{2224}{2}, \frac{2224}{4}, \frac{2224}{8}, \cdots, \frac{2224}{2^n} \}$ represents a geometric sequence.
        \item $S_n = \sum_{n=1}^{\infty} \frac{2224}{2^n}$ is a convergent geometric series since $\left|r\right| = \left|\frac{1}{2}\right| < 1$.
        \item Taylor Series: $\sum_{n=0}^{\infty} \frac{f^{(n)}(c)}{n!} (x - c)^n$
        \item Velocity Vector: $\vec{v}(t) = x'(t)\vec{i} + y'(t)\vec{j} = \left\langle \frac{dx}{dt}, \frac{dy}{dt} \right\rangle$
        \item Area of Polar Curve: $A = \frac{1}{2} \int_{\alpha}^{\beta} r^2 \,d\theta$
    \end{itemize}
\end{enumerate}
\newpage 
\begin{center}
\textbf{\LARGE{{Member 5:Moumita Modak}}} 
\end{center}
\begin{center}
    \Huge\textbf{CV}
\end{center}   
\pagenumbering{gobble}
% Custom colors
\definecolor{headerblue}{HTML}{1F3C88}
\definecolor{accent}{HTML}{007ACC}
\definecolor{darkgray}{gray}{0.2}

% Custom fonts

\renewcommand{\familydefault}{\sfdefault}

% Section formatting
\titleformat{\section}{\Large\bfseries\color{headerblue}}{}{0em}{}[\titlerule]
\titleformat{\subsection}{\large\bfseries\color{darkgray}}{}{0em}{}[\vspace{0.5em}\titlerule]

% Header
\pagestyle{fancy}
\fancyhf{}
\fancyhead[C]{\textcolor{headerblue}}

% Footer
\fancyfoot[C]{\thepage}
\renewcommand{\footrulewidth}{0.4pt}
\renewcommand{\headrulewidth}{0.4pt}



\begin{center}
    {\Huge \textbf{\textcolor{headerblue}{Moumita Modak}}}\\
    \vspace{2mm}
    \href{mailto:modakmoumita789@@gmail.com}{\textcolor{darkgray}{\faEnvelope \hspace{1mm} modakmoumita789@gmail.com}} \quad
    \href{tel:+918335920022}{\textcolor{darkgray}{\faPhone \hspace{1mm} +91-8335920022}} \\
    \href{https://github.com/moumitamodak}{\textcolor{darkgray}{\faGithub \hspace{1mm} github.com/moumitamodak}} \quad
    \href{https://www.linkedin.com/in/moumita-modak-1797062b9}{\textcolor{darkgray}{\faLinkedin \hspace{1mm} linkedin.com/in/moumita-modak-1797062b9}} \quad
\end{center}

\vspace{5mm}

\section*{Education}
\begin{tabular}{p{0.7\textwidth} p{0.3\textwidth}}
    \textbf{BSc in IT (Artificial Intelligence)}, Maulana Abul Kalam Azad University Of Technology, Kalyani & \hfill \textbf{2023 -- 2027} \\
    GPA: 8.0/10.0 & \\
\end{tabular}

\section*{Skills}
\begin{tabular}{p{0.5\textwidth} p{0.5\textwidth}}
    \textbf{Programming Languages:} & HTML, CSS, JavaScript, C, C++, Python \\
    \textbf{Web Development:} & React, Bootstrap, jQuery \\
    \textbf{Tools:} & Git, GitHub, VS Code, Postman \\
    \textbf{Design:} & Photoshop, Illustrator \\
\end{tabular}

\section*{Experience}
\subsection*{Web Development Projects}
\begin{itemize}[leftmargin=0.5cm]
    \item Developed responsive websites using HTML, CSS, JavaScript.
    \item Worked with peers on user-friendly interfaces.
    \item Integrated RESTful APIs in projects.
\end{itemize}

\subsection*{Project: SymbolApp}
\begin{itemize}[leftmargin=0.5cm]
    \item Created a Java AWT application to manage symbols.
    \item Implemented features like creation, deletion, modification.
\end{itemize}

\section*{Projects}
\subsection*{LaTeX Lab Notebook}
\begin{itemize}[leftmargin=0.5cm]
    \item Created a lab notebook using LaTeX with indexing and custom formatting.
\end{itemize}

\subsection*{Music Player}
\begin{itemize}[leftmargin=0.5cm]
    \item Developing a JavaScript music player with playlist management.
\end{itemize}


\section*{Interests}
\begin{itemize}[leftmargin=0.5cm]
    \item Gaming robotics,military applications,expert systems,search engines.
\end{itemize}
\end{document}
